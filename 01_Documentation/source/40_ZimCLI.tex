%\section{Command-line interface}
%\label{Sec:ComLine}

In this section is described a Command-Line Interface (CLI) for the
Zimulator. This CLI will be used in later sections in order to run
various example simulations, in addition to furnishing a complete
example implementation of the {\tt zio} interface.

\section{CLI source code}
The source-files are located in the {\tt ZimCLI} directory:
\begin{itemize}
\item {\tt CommandLine.java} -- class with {\tt main()} for command-line use; constructs {\tt Zimulation} class
\item {\tt Help.java} -- self-documentation for command-line
\item {\tt ComLineIO.java} -- implementation of {\tt zio} described in detail just below
\item {\tt FileOrHttpLineIterator.java} -- `translation' from a filename or http address to a string iterator
\end{itemize}

Within {\tt ComLineIO.java} the five {\tt zio} functionalities are implemented as follows:
\begin{itemize}
\item (1) Reading inputs:\\
  Filenames or http sources can be specified on the command-line; these are converted to the required string iterators.
\item (2) Consulting servers:\\
  The Server Identifiers in the specified \zobj{zsystem} are simply
  interpreted as http addresses. This is merely a choice made in the
  CLI; the identifiers could in fact be any strings at all.
\item (3) State storage:\\
  The streams for loading and saving of state are connected to files; the filenames are specified on the command line.
\item (4) Reporting:\\
  A filename is specified on the command line.
\item (5) Verbosity:\\
  This is simply sent to {\tt stdout}, with some attributes controlled by command-line switches.
\end{itemize}

\section{CLI usage}

Typical example usage for the examples which follow in this document is\\
\comline{java -jar Zimulator.jar zl=45 z=30 I=ZsyntaxSourceFile.zim R=output.zo }\\
This can be understood by running the CLI with no arguments, invoking the help feature and producing the terminal output:

\vspace{0.5cm}
%
% \input{TerminalOutput.tex}   % We choose the picture approach instead.
%
\includegraphics[angle=270,width=16cm]{40_figs/ComLineHelp.eps}
\vspace{0.5cm}

In the following \Chapref{Chap:SysExm} the CLI will be used to
run some simple example simulations.
