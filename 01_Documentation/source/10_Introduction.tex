
City transit systems often include some type of vehicle network which
is by design separated from general city traffic; in the present work,
such a network (or possibly a combination of such transit modes) will
be generically referred to as a `metro system'.  This term is used
somewhat loosely, and the considerations discussed will be variously
applicable to systems termed as Subway, LRT, Tram or Regional Rail
systems. The characteristics specifically required here are a lack of
coupling to street traffic, and (of course related) uniform behaviour
of vehicles within the system on a somewhat simpler network than that
of city streets. In fact, it is possible to address simulation of a
system with these characteristics without specifically speaking of a
metro system.

Given identification of these properties, it can be expected that the
microscopic modelling of such a system and its users might be
generalised in terms of some set of objects which interact via degrees
of freedum such as capacity, speed, origin-destination, and schedule.
In \Chapref{Chap:Zim} such a simulation system is described, first by
defining in an abstract way the dynamical objects and behaviour, and
then by reference to a concrete implementation. The discussion
in \Chapref{Chap:Zim} does not make explicit reference to trains,
vehicles or passengers; this avoidance of domain specificity is effected in
the hope that other systems which may have similar properties of being
locally described by one-dimensional progress and a simple set of
interaction rules might also be simulated. At the least, it is
expected that designing simulation technology around simple abstract
rules allows application to metro systems of different types and
construction.

In \Chapref{Chap:SysExm} a basic configuration of objects is assembled
to illustrate how a simple system can be realised and its behaviour
understood.

In \Chapref{Chap:Metro} some general details of metro-system
simulation are discussed, and it is shown how the objects defined in
\Chapref{Chap:Zim} can be assembled to form such a simulation. It is
indicated how generic data describing a metro system of interest may
be used to construct a model of such a system, and it is discussed how
details of interest like vehicle speeds, station layout, the metro
network itself and user demand map to the abstract objects in the
system.

In \Chapref{Chap:Madrid}, publicly-available data are used to
construct an approximate model of the metro system of Madrid, Spain,
with the intention of demonstrating a full example system. The
resultant model is used in subsequent chapters to illustrate
calibration and extraction of results.  In \Chapref{Chap:MadCali}, a
method is shown of calibrating the parameters of the passenger
walking-speed distribution in order to match real-world reference data
(`ground truth').  In \Chapref{Chap:MadResu}, a number of results are
extracted from a simulation run, demonstrating the system's use as a
digital twin, to access degrees of freedom which are inaccessible (or
practically inaccessible) in the real world.

The concrete implementation of the simulator is provided in terms of
Java source code, examples and auxiliary tools; indeed the primary
function of the present document is as comprehensive documentation of
this implementation. Programs and data relevant for each chapter are
generally contained in a directory beginning with the chapter number.
The material is hosted at:\\
\phref{https://github.ibm.com/sgresearch/Zimulator}{https://github.ibm.com/sgresearch/Zimulator}
