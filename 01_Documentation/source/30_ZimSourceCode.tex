

Section \secref{Sec:SFAS} is a high-level description of the source code
for the Zimulator. This section will be of little interest to the reader
only wishing to make use of the simulator without modification or
improvement.

Interfacing with the code (that is, making use of the Zimulator as a
library) is documented below in \Secref{Sec:zimzio}.
A complete example of interfacing with the Zimulator library is given in
\Secref{Chap:CLI} in the form of the command-line interface which
is used to run all the provided simulation examples.

\section{Source files and structure}
\label{Sec:SFAS}

Here are described in summary the Java source files which comprise the simulation core.
The following list is intended as a very high-level tour of the various source-code files;
of course an in-depth understanding of the source code can only be obtained by examination
of the code and associated comments, beginning with the {\tt zio} and {\tt zim} interfaces
and the {\tt Zimulation} class which is the public entry point.

\begin{itemize}
\item Public interfaces: (Discussed below in \Secref{Sec:zimzio})\\
  {\tt   zim.java}  -- implemented by the Zimulator\\
  {\tt   zio.java} -- implemented by the user (e.g.~CLI tool) to communicate with the Zimulator
\item Object file reading and writing:\\
  {\tt   CompiledFileIO.java} -- routines for reading and writing the system state\\
  {\tt   CompiledFileRW.java} -- interfaces for low-level IO of objects and reference resolution\\
  {\tt   PrimitiveDataIO.java} -- low-level routines for reading and writing basic types
\item Communication with external servers:\\
  {\tt   ExternalServer.java}  -- wrappers for communicating with servers\\
  {\tt   ServerFlags.java} -- routines for manipulation of the $\xi$ server flags specified\\
  {\tt   ServerRequest.java} -- serialization of server requests
\item Source-file zsyntax loading:\\
  {\tt   ZsyntaxInputParser.java} -- main parser for zsyntax input\\
  {\tt   LabelReferencing.java}  -- resolution of labels specified in zsyntax\\
  {\tt   LowLevelFileReading.java} -- low-level reading of zsyntax\\
  {\tt   ZtypeReferencing.java} -- resolution of $(A,n)$ for specified \zobj{ztypes}\\
  {\tt   ConnectivityResolution.java} -- consistency of \zobj{zlinks}, containment and \zobj{zpaths}
  {\tt   ConstantValues.java}
\item System and lists:\\
  {\tt   Zimulation.java} -- main simulation class; top-level control; {\tt zim} interface implementation\\
  {\tt   zsystem.java} -- holder for the system lists and other simulation parameters\\
  {\tt   SystemLists.java} -- the four system lists ($[0][S][T][Z]$; documented in the code)\\
  {\tt   Containment.java} -- management of containment of \zobj{zboxen}
\item Verbosity and Reporting:  \\
  {\tt Reporter.java} -- system output Reporting.
  {\tt Verbose.java} -- verbose output
  {\tt HTML\_output.java} -- output (debugging) of HTML-table representation of system \zobj{zbox} configuration\\
  {\tt PS\_output.java} -- output (debugging) of Postscript representation of system \zobj{zbox} configuration\\
  {\tt BoxCounter.java} -- counting of \zobj{zboxen} and production of live terminal output\\
  {\tt DoReportFlags.java} -- handling of various reporting flags\\
  {\tt ReportFlags.java} -- storage of reporting flags\\
  {\tt TimeRoutines.java} -- time syntax (e.g.~h:m:d)\\
  {\tt CommandLine.java} -- something to execute when Zimulator is invoked from command line
  (This only prints a message; it is \emph{not} the CLI described in \Chapref{Chap:CLI}.)
\item Object behaviour:\\
  {\tt  zobject.java},
  {\tt  zbox.java},
  {\tt  zdemand.java},
  {\tt  zpath.java},
  {\tt  zschedule.java},
  {\tt  zsource.java},
  {\tt  zstop.java},
  {\tt  ztype.java},
  {\tt  zlink.java} -- behaviour of the various \zobj{zobjects} described in \Chapref{Chap:Zim}\\
{\tt  Presence.java} -- fractional presence of \zobj{zboxen}; not utilised in metro-system simulation.\footnote{Thus, not well tested.}\\
{\tt  Accommodation.java} -- handling questions of capacity when \zobj{zboxen} shift to new containers
\item Paths and Links: \\
  {\tt   Economics.java} -- keeping track of expense of various portions of a \zobj{zpath}\\
  {\tt   ZboxNetwork.java} -- implementation of a Network interface from the auxiliary library, used to assign \zobj{zpaths}\\
  {\tt   PathResolution.java} -- filling in missing \zobj{zboxen} along \zobj{zpaths}\\
  {\tt   Route.java} -- keeping track of logistics of \zobj{zpaths}\\
  {\tt   Neighbourhoods.java} -- storage of various $\Lambda$ neighbourhoods (described in \Secref{lambdaneigh})\\
  {\tt   NeighbourhoodCalculation.java} -- calculation of the $\Lambda$ neighbourhoods 
\end{itemize}

\section{{\tt zim} and {\tt zio} interfaces}
\label{Sec:zimzio}

\emph{The Zimulator implements {\tt zim}; the user implements {\tt zio}.}
This is explained in the present section.

The {\tt zio} interface is very simple:
\begin{lstlisting}[mathescape]
public interface zim
{
    public void adjustFlags(String[] VarValPairs);   
    public void adjustParameters(String[] VarValPairs);
    public int run();
}
\end{lstlisting}
The Zimulator is intended to be used by performing the following
steps. A concrete example of doing so can be found in the CLI
described in \Chapref{Chap:CLI}.
\begin{itemize}
\item  Instantiate a class which implements the {\tt Zimulator.zio} interface (described just below). \\
  This class describes various input and output streams.
\item  Instantiate a Zimulation class; {\tt MyZim = new Zimulation(zio MyZio)}. This will contain default settings.
\item  Optionally call {\tt MyZim.adjustFlags()} and {\tt MyZim.adjustParameters()} to control the simulation.
  Arguments to these methods contain {\tt variable=value} pairs, as used with the CLI tool, described in \Chapref{Chap:CLI}.
\item Call {\tt MyZim.run()} to perform the simulation. \\
  This initiates actual simulation. It does not return until the simulation has finished.
  During the simulation, of course, servers might be consulted, depending on how the system has been prepared.
\item The output from the simulation will be sent, \emph{as it is simulated}, in accordance with that specified in {\tt zio}.
\end{itemize}

As mentioned, all of this requires the user to implement the five functionalities (six methods) of the {\tt zio} interface:
\begin{lstlisting}[mathescape]
public interface zio
{
  // (1) Reading inputs:
  Iterator<Iterator<String>> getZsyntaxInputs();
  // (2) Consulting servers:
  Iterator<String> makeZserverRequest(String ServerId, Iterator<String> Request);
  // (3) State storage:
  DataOutputStream saveZobjFile();
  DataInputStream loadZobjFile();
  // (4) Reporting:
  void sendReportLine(int ReportChannel,String ReportLine);
  // (5) Verbosity:
  void sendVerboseLine(String VerboseLine);
}
\end{lstlisting}

This interface is designed in such a way as to provide complete versatility for the user.
For a concrete implementation of the {\tt zio} interface, the reader is referred to \Chapref{Chap:CLI}.

\begin{itemize}
\item {\tt getZsyntaxInputs()} \\
  By means of this iterator, the Zimulator reads in initial input zsyntax.
  The form of `iterator over iterators of strings' is meant to make it trivial to feed the Zimulator a number of source files.
\item {\tt makeZserverRequest()} \\
  The {\tt ServerIdentifier} is merely the identifier text string provided in the \zobj{zsystem} $\Xi$ list.
  The reply from such a server should be zsyntax text.
\item {\tt saveZobjFile()} and {\tt loadZobjFile()} \\
  These methods are called to store and retrieve the system state using an internal object-file format.\\
  Of course, the user might elect that this be done via files or in some other way.
\item {\tt sendReportLine()} \\
  All reporting output from the simulation is sent through this function; a line at a time is sent,
  with a channel specified. The channel number used is that specified in the $R$ field of the object making the report.
\item {\tt sendVerboseLine()} \\
  All verbosity output from the simulation is sent via this function, one line at a time.
\end{itemize}
It is guaranteed that all provided iterators will be used immediately and completely.
DataInputStream and DataOutputStream for object files are also accessed ``all at once''
and subsequently closed.

The difference between \emph{verbosity} and \emph{reporting} has been described in \Secref{secrep}.

It is noted that a `server' is only external from the point of view of the core Zimulator.
It may indeed be implemented as a `slow' process which replies to an http request on a distant machine,
or it may be implemented as `fast' local code. Therefore, there is little intrinsic performance penalty
in delegating various non-core functionality to zservers. The only requirement is that whatever
server is implemented must emit zsyntax replies for ingestion by the Zimulator core.

